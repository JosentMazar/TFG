\chapter{Conclusiones}

Es difícil de ignorar tras finalizar el estudio de estas muestras de malware que, pese a la variedad de métodos y vectores de ataque, el más explotado es la ignorancia del usuario. Y no es de extrañar que esto sea así.\\
La tecnología evoluciona cada vez más rápidamente, y a menudo dependemos más y más de ella, integrándola en cada aspecto de nuestras vidas. Desde teléfonos móviles que contienen identificaciones biométricas y contraseñas bancarias, a casas inteligentes cuyas puertas pueden ser controladas desde un portal web. Un desarrollador o un ingeniero puede tratar de crear el candado perfecto (Que aun así, nunca lo sera), pero no puede evitar que abras la puerta e invites al ladrón a pasar.\\
Tener un cuerpo de empleados formados en al menos las técnicas de phishing, que sean capaces de reconocer un mail falso o cuando un archivo es malicioso es probablemente el mejor paso que se puede tomar para evitar ser victimas de este tipo de ataques, pero al fin y al cabo solo somos humanos, e incluso el ingeniero más preparado puede estar distraído, hacer click en el enlace equivocado, o confundir el enlace a la pagina de login de un sitio web por otro al que se le ha cambiado una letra de sitio. Y en muchas ocasiones, un despiste, un error o un mal día es todo lo que se necesita para vulnerar todo un sistema.\\\\
Y es por eso mismo, que es importante estudiar y enfrentar los ataques y programas maliciosos desde un punto de vista técnico. El factor humano puede ser la mayor vulnerabilidad de un sistema, pero es factor que por mucho que pueda mitigarse, no dejara de existir nunca.\\
Estudiar y reconocer las vulnerabilidades más explotadas, como aplicaciones desactualizadas que presentan potenciales vectores de entrada para atacantes, o maquinas mal configuradas es otro gran paso que se puede dar para mantener un sistema seguro y reducir al máximo las posibilidades de sufrir un ataque, y estos son pasos que un administrador de sistemas o un ingeniero pueden tomar para mantener seguro un sistema pese al factor humano.\\\\
Gran parte de la lucha contra el malware en particular, se basa en la capacidad de análisis. No por nada se intenta por todos los medios que el código de un virus este lo más oculto posible y se a lo más difícil de identificar, incluso cuando el "malware" en si no es mas que un macro que ejecuta algunas lineas por consola.\\
Cuanto más complejo es un malware, más esfuerzo se le ha puesto en ocultar su funcionamiento, y más técnicas para impedir su análisis se han utilizado. Al fin y al cabo, ya sea aprendiendo a reconocer ataques de phishing, a mantener actualizado y correctamente configurado un equipo conectado a la red, o estudiando el código en ensamblador de un ejecutable, no existe mejor arma contra las amenazas en la red que la capacidad de aprendizaje y análisis para aprender de los propios atacantes.